\documentclass[fleqn]{article}
\usepackage[left=1in, right=1in, top=1in, bottom=1in]{geometry}
\usepackage{mathexam}
\usepackage{verbatim}

\ExamClass{CSCE 240}
\ExamName{Homework 1}
\ExamHead{Due: Wed 11 Sep 2019}

\let
\ds
\displaystyle

\begin{document}
\ExamInstrBox {
  You shall submit a zipped, \textbf{and only zipped}, archive of your homework
  directory, hw1. The directory shall contain, at a minimum, the file
  \texttt{base\_decomposer.cc}. Your submission file must be named hw1.zip. \\

  I will use my own makefile to make your \texttt{base\_decomposer.cc} file. Do
  not use a header for this assignment. My grader will not look for one.
}
\section*{Introduction}
%
The most common representation of quantities in base-10 is the power series
positional counting system. In this system, 123,456 represents the power series \\
%
\begin{center}
  $1\times10^5+2\times10^4+3\times10^3+4\times10^2+5\times10^1+6\times10^0$
\end{center}
\begin{center}
  \textbf{OR}
\end{center}
\begin{center}
  $100000+20000+3000+400+50+6$
\end{center}
The same value presented in negative magnitude -123,456, might be
\begin{center}
  $-(100000+20000+3000+400+50+6)$
\end{center}
%
%
\section*{Description}
Develop a small application for me which will produce output in the second
form--that is:
\begin{center}
  $100000+20000+3000+400+50+6$
\end{center}
\begin{center}
  \textbf{OR}
\end{center}
\begin{center}
  $-(100000+20000+3000+400+50+6)$
\end{center}
%
Without prompting, read input from STDIN as a signed integer value. Extract each
place of the integer and print the conversion to STDOUT. \\
%
I provided you with a test file to test your code. You should ensure that your
code satisfies the tester's requirements. It is the tool I will use to grade
your submissions. I will only change the input and expected values. \\
%
To utilize the tester, you will need access to a python3 interpreter. The tester
can be called as follows, assuming that \texttt{python3} is in your path and
that your present working directory is \texttt{../hw1}
  \begin{verbatim}
python3 test_decomposer 1
python3 test_decomposer 2
  \end{verbatim}
%
%
\section*{Point Awards}
Test 1 is worth 2 of 3 points and Test 2 is worth 1 of 3 points.
\vspace{1.0em}
%
I have provided you a make file. You should definitely read the
makefile and I would encourage you to read the python tester.\\
\\
%
Late assignments will lose 25\% per day late, with no assignment begin accepted
after 4 days (100\% reduction in points).\\
\\
%
Check your syllabus for the breakdown of grading.
\end{document}


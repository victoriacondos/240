\documentclass[fleqn]{article}
\usepackage[left=1in, right=1in, top=1in, bottom=1in]{geometry}
\usepackage{mathexam}
\usepackage{verbatim}

\ExamClass{CSCE 240}
\ExamName{Homework 5}
\ExamHead{Due: 18 Nov 2019}

\let
\ds
\displaystyle

\begin{document}
\ExamInstrBox {
  You shall submit a zipped, \textbf{and only zipped}, archive of your homework
  directory, hw5. The directory shall contain, at a minimum, the files
  \texttt{inc/point.h} and \texttt{src/point.cc}. Name the archive submission
  file \textbf{hw5.zip} \\

  I will use my own makefile to compile and link to your \texttt{inc/point.h}
  and \texttt{src/point.cc} files. You must submit, at least, these two files.
}
%
You must provide the following public interface for your class:
\begin{enumerate}
  \item \texttt{Point::\underline{CalcBoundingBox}}\footnote{
      This method must be a static method}
  \item \texttt{Point::Point()}
  \item \texttt{Point::Point(double, double)}
  \item \texttt{Point::Point(double, double, double)}
  \item \texttt{Point::Point(const double[], size\_t)}
  \item \texttt{Point::Point(const Point\&)}
  \item \texttt{Point::\textasciitilde Point()}
  \item \texttt{Point::operator=(const Point\&)}
  \item \texttt{Point::operator[](const Point\&)}
  \item \texttt{Point::operator==(const Point\&)}
\end{enumerate}
Read the provided header file documentation for instructions on how the
methods should behave. You may use any member or method in your private
class section.\\
%
\\
I have provided you a set of basic test apps which you can use to ensure that
your code is, at least partially, correct. As always, I would suggest a more
rigorous testing scheme to ensure that your methods handle missing data. \\
%
\\
Late assignments will lose 25\% per day late, with no assignment begin accepted
after 3 days (at 100\% reduction in points).\\
\\
%
You will receive
\begin{itemize}
  \item 3 points for correct constructors,
  \item 1 point for correct assignment op,
  \item 1 point for correct bracket op,
  \item 1 point for correct equal to op,
  \item 2 points for correct CalcBoundingBox method
\end{itemize}
\end{document}

